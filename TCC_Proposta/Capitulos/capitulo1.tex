%%%%%%%%%%%%%%%%%%%%%%%%%%%%%%%%%%%%%%%%%%%%%%%%%%%%%%%%%%%%%%%%%%%%%%%%%%%%%%%
% CAP�TULO 1
\chapter{Introdu��o}  
Amplamente  debatido, o tema da qualidade de energia tem enorme import�ncia nos dias atuais. Com processos industriais cada vez mais automatizados, a opera��o eficiente e o controle das m�quinas se torna gradativamente mais dependente da qualidade da energia el�trica. 

 Diversas defini��es podem ser adotadas para definir o que se entende como qualidade de energia. Tomando o ponto de vista do consumidor como o mais importante, Dugan define o tema como qualquer problema manifestado em desvios na corrente, tens�o ou frequ�ncia que resultem em falha ou mau funcionamento do equipamento do cliente \cite{dugan2002electrical} 
\section{Problema}


\section{Justificativa}


\section{Objetivos}
O objetivo deste trabalho � o desenvolvimento de um sistema de detec��o de varia��es de tens�o de curta 
dura��o em redes de distribui��o de energia el�trica, mantendo as informa��es relativas a cada 
ocorr�ncia dispon�veis para consulta online por concession�rias de energia.

Definido o objetivo geral do trabalho, pode-se destacar os seguintes pontos como objetivos espec�ficos:

\begin{itemize}
\item [-]Configurar o microcontrolador PIC32MX795F512L para realizar a comunica��o com o circuito integrado ADE7758;
\item [-]Programar o circuito integrado ADE7758 para realizar a detec��o dos diferentes tipos de varia��es de tens�o de curta dura��o;
\item [-]Configurar o microcontrolador PIC32MX795F512L para funcionar como um servidor, salvando as informa��es das ocorr�ncias em tempo real.
\end{itemize}

\section{Organiza��o do Texto}
